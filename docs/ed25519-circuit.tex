\documentclass[a4paper, 12pt]{article}
\usepackage[margin=0.5in]{geometry}
\usepackage{amsmath, amssymb}
\usepackage{hyperref}

\title{Ed25519 Circuits}
\author{Saravanan Vijayakumaran}

\begin{document}
\maketitle

\section{Ed25519 Elliptic Curve Equations}%
\label{sec:curve_equations}
Let $q$ be the prime $2^{255}-19$. The ed25519 curve is given by the set
\begin{align}
  E = \left\{ (x,y) \in \mathbb{F}_q \times \mathbb{F}_q \mid -x^2+y^2=1+dx^2y^2 \right\},
  \label{eqn:curve}
\end{align}
where $d = -\frac{121665}{121666}$. The addition law is given by
\begin{align}
  (x_3, y_3) = (x_1, y_1) + (x_2, y_2) = \left( \frac{x_1y_2+x_2y_1}{1+dx_1x_2y_1y_2}, \frac{y_1y_2 + x_1x_2}{1-dx_1x_2y_1y_2}  \right).
  \label{eqn:addition}
\end{align}
The identity element is the point $(0,1)$. The above addition law works even when the points being added are identical, i.e.~$(x_1, y_1) = (x_2, y_2)$. So we do not require a different addition rule for point doubling.

\section{Verifying Point Addition}%
\label{sec:verifying_point_addition}
In the context of SNARKs, we are concerned with verifying that three points $P, Q, R \in E$ satisfy $R= P+Q$ using an arithmetic circuit. The main challenge is the representation of the arithmetic over $\mathbb{F}_q$ using field elements from a different finite field $\mathbb{F}_n$. The field $\mathbb{F}_n$ is called the \textit{native field} and the arithmetic over $\mathbb{F}_q$ is called \textit{non-native arithmetic}.

A few popular choices for the native field have $r$ equal to a 254-bit prime. The \textit{capacity} of such fields is then 253 bits, i.e.~they can represent any integer in the range $\{0,1,\ldots,2^{253}-1\}$. In this document, we exclusively focus on this case.

An element of $\mathbb{F}_q$ requires 255 bits and hence cannot be expressed a single element of $\mathbb{F}_n$. While two elements of $\mathbb{F}_n$ suffice to represent an element of $\mathbb{F}_q$, we will use more than two to allow the representations of products of $\mathbb{F}_q$ elements.

Suppose we use four elements of $\mathbb{F}_n$ to represent an element of $\mathbb{F}_q$. Let $a \in \mathbb{F}_q$ and $a_0, a_1, a_2, a_3 \in \mathbb{F}_n$ such that
\begin{align*}
  a = \sum^{3}_{i=0} a_i 2^{64i}.
\end{align*}
In the \textit{reduced representation} of $a$, the $a_i$'s are in the range  $\{0,1,2,\ldots,2^{64}-1\}$. The elements $a_i$ are called the \textit{limbs} of $a$.

Arithmetic operations can lead to \textit{unreduced representations}. Let $a=\sum_{i=0}^{3} a_i 2^{64i}, b=\sum_{i=0}^{3} b_i 2^{64i}$ for $a_i, b_i \in \mathbb{F}_{r} \cap \{0,1,\ldots,2^{64}-1\}$. The product of $a$ and $b$ is given by
\begin{align*}
  ab = \sum^{6}_{k=0} c_k 2^{64k}  \ \ \ \ \ \ \text{ where } c_k = \underset{i+j=k}{\sum_{i=0}^3 \sum_{j=0}^3} a_i b_j.
\end{align*}
Each product $a_ib_j$ occupies 128 bits and the sum $c_k$ can occupy a maximum of $128 + 2 = 130$ bits. As $\mathbb{F}_n$ has a capacity of 253 bits, we can represent the product $ab$ using seven $\mathbb{F}_n$ elements $c_0,c_1,\ldots,c_6$. This would be an unreduced representation of the product.

It may be possible to work with unreduced representations and delay reducing them to obtain smaller arithmetic circuits.\footnote{The \texttt{circom-ecdsa} project uses this approach. \url{https://github.com/0xPARC/circom-ecdsa}} But we should be careful to avoid overflow in the limbs. For example, the unreduced representation of $\mathbb{F}_q$ elements using $\mathbb{F}_n$ limbs encounters overflow in a degree 4 product, i.e.~an expression of the form $x_1 x_2 x_3 x_4$ where each $x_i \in \mathbb{F}_q$. Each term of such a product will require at least 256 bits.

To verify that points $(x_1, y_1), (x_2, y_2), (x_3, y_3)$ satisfy the addition law in (\ref{eqn:addition}), we could check that
\begin{align*}
  x_3(1+dx_1x_2y_1y_2) &= x_1y_2 + x_2y_1,\\
  y_3(1-dx_1x_2y_1y_2) &= x_1x_2 + y_1y_2.
\end{align*}
The above equations involve degree 6 products. These equations use affine coordinates.

Using projective coordinates\footnote{\url{https://hyperelliptic.org/EFD/g1p/auto-twisted-projective.html}}, the point addition formula is given by
\begin{align*}
  C   & = X_1X_2,\\
  D   & = Y_1Y_2,\\
  E   & = dCD,\\
  X_3  & = (1-E)((X_1+Y_1)(X_2+Y_2)-C-D),\\
  Y_3  & = (1+E)(D+C),\\
  Z_3  & = 1-E^2.
\end{align*}
These also involve a degree 6 product (in $Z_3$) and several degree 5 products.

Using extended coordinates\footnote{\url{https://hyperelliptic.org/EFD/g1p/auto-twisted-extended-1.html}}, the point addition formula is given by
\begin{align*}
  A & = (Y_1-X_1)(Y_2-X_2),\\
  B & = (Y_1+X_1)(Y_2+X_2),\\
  C & = kT_1T_2,\\
  D & = 2Z_1Z_2,\\
  E & = B-A,\\
  F & = D-C,\\
  G & = D+C,\\
  H & = B+A,\\
  X_3 & = EF,\\
  Y_3 & = GH,\\
  T_3 & = EH,\\
  Z_3 & = FG,
\end{align*}
where $k=2d$. These also involve a degree 6 product (in $Z_3$) and several degree 5 products. The inverted coordinates also involve degree 6 products.\footnote{\url{https://hyperelliptic.org/EFD/g1p/auto-twisted-inverted.html}}

There does not seem any way to reduce the degree of the products in the point addition verification equations to 3 without resorting to intermediate reductions. So we cannot use 64-bit $\mathbb{F}_n$ limbs to represent $\mathbb{F}_q$ elements \textit{if we want to restrict ourselves to a single unreduced to reduced representation conversion}.

The maximum limb bitwidth such that a product of six limbs fits in 253 bits is 42, as $6\times 42 = 252$. But this will not accommodate carries.

In general, the product between $a = \sum_{i=0}^{k_a-1}a_i 2^{\eta i}$ and $b= \sum_{i=0}^{k_b-1}b_i 2^{\eta i}$ will have $k_a+k_b-1$ limbs each having a maximum bitwidth of $m_a+m_b+\left\lceil \log_2 \left( \max\left( k_a, k_b \right) \right) \right\rceil$ where the $a_i$'s have bitwidth $m_a$ and the $b_i$'s have bitwidth $m_b$.

We would need 7 limbs each having bitwidth equal to 42 to represent an $\mathbb{F}_q$ element (as $6\times 42 = 252 < 255 < 294 = 7\times 42$). Setting $k_a=k_b =7$ shows that even a degree 2 product will require 3 bits to accomodate carries. A degree 6 product of $\mathbb{F}_q$ elements represented using 42-bit limbs from $\mathbb{F}_n$ will have overflow in its unreduced representation.

Let us now consider the case of 41-bit limbs. We again need 7 limbs to represent an $\mathbb{F}_q$ element as $255 < 287 = 7\times 41$. A degree 6 product of 7-limbed terms with 41 bits per limb will need 268 bits to products and carries in the unreduced representation. The calculations are illustrated in Table \ref{tab:limb41}.
\begin{table}[h]
  \centering
  \begin{tabular}{c|c|c|c|c|c|c}
    $a$ & $b$ & $m_a$ & $m_b$ & $k_a$ & $k_b$ & $m_a+m_b+\left\lceil \log_2 \left( \max\left( k_a, k_b \right) \right) \right\rceil$ \\ \hline
    $z_1$ & $z_2$ & 41 & 41 & 7 & 7 & 85 \\ \hline
    $z_1z_2$ & $z_3$ & 85 & 41 & 13 & 7 & 130 \\ \hline
    $z_1z_2z_3$ & $z_4$ & 130 & 41 & 19 & 7 & 176 \\ \hline
    $z_1z_2z_3z_4$ & $z_5$ & 176 & 41 & 25 & 7 & 222 \\ \hline
    $z_1z_2z_3z_4z_5$ & $z_6$ & 222 & 41 & 31 & 7 & 268 \\ \hline
  \end{tabular}
  \caption{Bitwidth growth for products of terms with 7 limbs of 41 bits each}
  \label{tab:limb41}
\end{table}

A degree 6 product of 7-limbed terms with 40 bits per limb will need 262 bits to products and carries in the unreduced representation. The calculations are illustrated in Table \ref{tab:limb40}.
\begin{table}[h]
  \centering
  \begin{tabular}{c|c|c|c|c|c|c}
    $a$ & $b$ & $m_a$ & $m_b$ & $k_a$ & $k_b$ & $m_a+m_b+\left\lceil \log_2 \left( \max\left( k_a, k_b \right) \right) \right\rceil$ \\ \hline
    $z_1$ & $z_2$ & 40 & 40 & 7 & 7 & 83 \\ \hline
    $z_1z_2$ & $z_3$ & 83 & 40 & 13 & 7 & 127 \\ \hline
    $z_1z_2z_3$ & $z_4$ & 127 & 40 & 19 & 7 & 172 \\ \hline
    $z_1z_2z_3z_4$ & $z_5$ & 172 & 40 & 25 & 7 & 217 \\ \hline
    $z_1z_2z_3z_4z_5$ & $z_6$ & 217 & 40 & 31 & 7 & 262 \\ \hline
  \end{tabular}
  \caption{Bitwidth growth for products of terms with 7 limbs of 40 bits each}
  \label{tab:limb40}
\end{table}

A degree 6 product of 7-limbed terms with 39 bits per limb will need 256 bits to products and carries in the unreduced representation. The calculations are illustrated in Table \ref{tab:limb39}.
\begin{table}[h]
  \centering
  \begin{tabular}{c|c|c|c|c|c|c}
    $a$ & $b$ & $m_a$ & $m_b$ & $k_a$ & $k_b$ & $m_a+m_b+\left\lceil \log_2 \left( \max\left( k_a, k_b \right) \right) \right\rceil$ \\ \hline
    $z_1$ & $z_2$ & 39 & 39 & 7 & 7 & 81 \\ \hline
    $z_1z_2$ & $z_3$ & 81 & 39 & 13 & 7 & 124 \\ \hline
    $z_1z_2z_3$ & $z_4$ & 124 & 39 & 19 & 7 & 168 \\ \hline
    $z_1z_2z_3z_4$ & $z_5$ & 168 & 39 & 25 & 7 & 212 \\ \hline
    $z_1z_2z_3z_4z_5$ & $z_6$ & 212 & 39 & 31 & 7 & 256 \\ \hline
  \end{tabular}
  \caption{Bitwidth growth for products of terms with 7 limbs of 39 bits each}
  \label{tab:limb39}
\end{table}

A degree 6 product of 7-limbed terms with 38 bits per limb will need 261 bits to products and carries in the unreduced representation. The calculations are illustrated in Table \ref{tab:limb38}.
\begin{table}[h]
  \centering
  \begin{tabular}{c|c|c|c|c|c|c}
    $a$ & $b$ & $m_a$ & $m_b$ & $k_a$ & $k_b$ & $m_a+m_b+\left\lceil \log_2 \left( \max\left( k_a, k_b \right) \right) \right\rceil$ \\ \hline
    $z_1$ & $z_2$ & 38 & 38 & 7 & 7 & 79 \\ \hline
    $z_1z_2$ & $z_3$ & 79 & 38 & 13 & 7 & 121 \\ \hline
    $z_1z_2z_3$ & $z_4$ & 121 & 38 & 19 & 7 & 164 \\ \hline
    $z_1z_2z_3z_4$ & $z_5$ & 164 & 38 & 25 & 7 & 207 \\ \hline
    $z_1z_2z_3z_4z_5$ & $z_6$ & 207 & 38 & 31 & 7 & 250 \\ \hline
  \end{tabular}
  \caption{Bitwidth growth for products of terms with 7 limbs of 38 bits each}
  \label{tab:limb38}
\end{table}

While it is possible to use 38-bit limbs to safely calculate the unreduced representation of a degree 6 product, the circuit logic can be quite complex. It seems prudent to try a simpler approach first, even if it is inefficient in terms of arithmetic circuit size.

\newpage
\subsection{An Approach using Four 64-bit Limbs}%
\label{subsec:an_approach_using_four_64_bit_limbs}
Unreduced representations of cubic products using four 64-bit limbs can be safely calculated in fields with capacity 253 bits. The bitwidth growth of this case is illustrated in Table \ref{tab:limb64}.
\begin{table}[h]
  \centering
  \begin{tabular}{c|c|c|c|c|c|c}
    $a$ & $b$ & $m_a$ & $m_b$ & $k_a$ & $k_b$ & $m_a+m_b+\left\lceil \log_2 \left( \max\left( k_a, k_b \right) \right) \right\rceil$ \\ \hline
    $z_1$ & $z_2$ & 64 & 64 & 4 & 4 & 130 \\ \hline
    $z_1z_2$ & $z_3$ & 130 & 64 & 7 & 4 & 197 \\ \hline
  \end{tabular}
  \caption{Bitwidth growth for products of terms with 7 limbs of 38 bits each}
  \label{tab:limb64}
\end{table}

Recall that the affine point addition verification equations can be written as
\begin{align}
  x_3(1+dx_1x_2y_1y_2) &= x_1y_2 + x_2y_1,\\
  y_3(1-dx_1x_2y_1y_2) &= x_1x_2 + y_1y_2.
  \label{eqn:additionAffineVerification}
\end{align}
We propose to try the following approach. It is inspired by the \texttt{circom-ecdsa} approach for verifying secp256k1 point addition.
\begin{itemize}
  \item Calculate the reduced representation of the cubic $dx_1x_2$. Let this representation be given by $u = \sum_{i=0}^{3}u_i 2^{64i}$.
  \item Calculate the reduced representation of the cubic $uy_1y_2$. Let this representation be given by $v = \sum_{i=0}^{3}v_i 2^{64i}$. Note that $v$ corresponds to the reduced representation of $dx_1x_2y_1y_2$.
  \item Check that the following quadratic equations hold.
    \begin{align*}
      x_3 + x_3 v - x_1y_2 - x_2y_1 = 0,\\
      y_3 - y_3 v - x_1x_2 - y_1y_2 = 0.\\
    \end{align*}
\end{itemize}

\section{Reducing a Cubic Product}%
\label{sec:reducing_a_cubic_product}
Let $a,b,c$ be three elements in $\mathbb{F}_q$ available in their reduced representations.
\begin{align*}
  a = \sum^{3}_{i=0} a_i 2^{64i}, \ \ \ \ b = \sum^{3}_{i=0} b_i 2^{64i}, \ \ \ \ c = \sum^{3}_{i=0} c_i 2^{64i}.
\end{align*}
Let $f = ab$. Then we have $f = \sum_{i=0}^{6} f_i 2^{64i}$ where
\begin{align*}
  f_0 & = a_0b_0,\\
  f_1 & = a_0b_1 + a_1b_0,\\
  f_2 & = a_0b_2 + a_1b_1 + a_2b_0,\\
  f_3 & = a_0b_3 + a_1b_2 + a_2b_1 + a_3b_0,\\
  f_4 & = a_1b_3 + a_2b_2 + a_3b_1,\\
  f_5 & = a_2b_3 + a_3b_2,\\
  f_6 & = a_3b_3.
\end{align*}
The limb $f_3$ occupies a maximum of 130 bits and the other limbs occupy 128 or 129 bits.

Let $g = fc$. Then we have $g = \sum_{i=0}^{9} g_i 2^{64i}$ where
\begin{align*}
  g_0 & = f_0c_0,\\
  g_1 & = f_0c_1 + f_1c_0,\\
  g_2 & = f_0c_2 + f_1c_1 + f_2c_0,\\
  g_3 & = f_0c_3 + f_1c_2 + f_2c_1 + f_3c_0,\\
  g_4 & = f_1c_3 + f_2c_2 + f_3c_1 + f_4c_0,\\
  g_5 & = f_2c_3 + f_3c_2 + f_4c_1 + f_5c_0,\\
  g_6 & = f_3c_3 + f_4c_2 + f_5c_1 + f_6c_0,\\
  g_7 & = f_4c_3 + f_5c_2 + f_6c_1,\\
  g_8 & = f_5c_3 + f_6c_2,\\
  g_9 & = f_6c_3,\\
\end{align*}
In terms of $a_i,b_i,c_i$, the limbs of $g$ are given by
\begin{align*}
  g_0 & = a_0b_0c_0,\\
  g_1 & = a_0b_0c_1 + a_0b_1c_0 + a_1b_0c_0,\\
  g_2 & = a_0b_0c_2 + a_0b_1c_1 + a_1b_0c_1 + a_0b_2c_0 + a_1b_1c_0 + a_2b_0c_0,\\
  g_3 & = a_0b_0c_3 + a_0b_1c_2 + a_1b_0c_2 + a_0b_2c_1 + a_1b_1c_1 + a_2b_0c_1 + a_0b_3c_0 + a_1b_2c_0 + a_2b_1c_0 + a_3b_0c_0,\\
  g_4 & = a_0b_1c_3 + a_1b_0c_3 + a_0b_2c_2 + a_1b_1c_2 + a_2b_0c_2 + a_0b_3c_1 + a_1b_2c_1 + a_2b_1c_1 + a_3b_0c_1 + a_1b_3c_0 + a_2b_2c_0 + a_3b_1c_0,\\
  g_5 & = a_0b_2c_3 + a_1b_1c_3 + a_2b_0c_3 + a_0b_3c_2 + a_1b_2c_2 + a_2b_1c_2 + a_3b_0c_2 + a_1b_3c_1 + a_2b_2c_1 + a_3b_1c_1 + a_2b_3c_0 + a_3b_2c_0,\\
  g_6 & = a_0b_3c_3 + a_1b_2c_3 + a_2b_1c_3 + a_3b_0c_3 + a_1b_3c_2 + a_2b_2c_2 + a_3b_1c_2 + a_2b_3c_1 + a_3b_2c_1 + a_3b_3c_0,\\
  g_7 & = a_1b_3c_3 + a_2b_2c_3 + a_3b_1c_3 + a_2b_3c_2 + a_3b_2c_2 + a_3b_3c_1,\\
  g_8 & = a_2b_3c_3 + a_3b_2c_3 + a_3b_3c_2,\\
  g_9 & = a_3b_3c_3,\\
\end{align*}
The limbs $g_4$ and $g_5$ occupy a maximum of 195 bits, 192 bits for the products and 3 bits for the carry from the addition of 12 products. The other limbs occupy fewer than 195 bits.

Since $q = 2^{255}-19$, we have $2^{256} = 38 \bmod q$ and $2^{512} = 38^2 = 1444 \bmod q$. We can rewrite $g$ as
\begin{align*}
  g  = \sum^{9}_{i=0} g_i 2^{64i} &= g_0 + g_1 2^{64} + g_2 2^{128} + g_3 2^{192} + g_4 2^{256} +   g_5 2^{320} + g_6 2^{384} + g_7 2^{448} + g_8 2^{512} + g_9 2^{576}\\
   &= g_0 + g_1 2^{64} + g_2 2^{128} + g_3 2^{192} + 2^{256} \left(  g_4 +   g_5 2^{64} + g_6 2^{128} + g_7 2^{192} \right) + 2^{512}\left(g_8  + g_9 2^{64} \right)\\
   &= g_0 + g_1 2^{64} + g_2 2^{128} + g_3 2^{192} + 38 \left(  g_4 +   g_5 2^{64} + g_6 2^{128} + g_7 2^{192} \right) + 1444 \left(g_8  + g_9 2^{64} \right)\\
   &= g_0 + 38 g_4 + 1444 g_8 + 2^{64} \left( g_1 + 38g_5 +1444g_9 \right)+ 2^{128} \left(  g_2 + 38 g_6\right) + 2^{192} \left( g_3 +38 g_7\right)\\
   & = h_0 + h_1 2^{64} + h_2 2^{128} + h_3 2^{192},
\end{align*}
where
\begin{align*}
  h_0 & = g_0 + 38 g_4 + 1444 g_8,\\
  h_1 & = g_1 + 38 g_5 + 1444 g_9,\\
  h_2 & = g_2 + 38 g_6,\\
  h_3 & = g_3 + 38 g_7.\\
\end{align*}
We note the following:
\begin{itemize}
  \item Each of the $g_i$'s occupy a maximum of 192 bits.
  \item The numbers 38 and 1444 occupy 6 bits and 11 bits respectively.
  \item $1444g_8$ occupies a maximum of 203 bits. This implies $h_0$ can occupy a maximum of 205 bits.
  \item By the same argument, $h_1$ occupies a maximum of 205 bits.
  \item $38g_6$ occupies a maximum of 198 bits. This implies $h_2$ can occupy a maximum of 199 bits.
  \item By the same argument, $h_3$ occupies a maximum of 199 bits.
  \item So all the $h_i$'s can fit in $\mathbb{F}_n$ limbs.
  \item The maximum value of $g$ is bounded by
    \begin{align*}
      2^{205} - 1 + \left(2^{205} - 1\right) 2^{64}+ \left( 2^{199}-1 \right)2^{128}+ \left( 2^{199}-1 \right)2^{192} < 2^{205} + 2^{269} + 2^{327} + 2^{391} < 2^{392}
    \end{align*}
  \item So the reduced representation of $g$ will have at most seven 64-bit limbs. Let $g_0', g_1',\ldots,g_6'$ denote these limbs.
   \begin{align*}
   g & = h_0 + h_1 2^{64} + h_2 2^{128} + h_3 2^{192} = \sum^{6}_{i=0} g_i' 2^{64i}\\
     & = g'_0 + g'_1 2^{64} + g'_2 2^{128} + g'_3 2^{192} + g'_4 2^{256} +   g'_5 2^{320} + g'_6 2^{384}
   \end{align*}
\end{itemize}

We want to find an $r \in \mathbb{F}_q$ in reduced representation such that 
\begin{align}
  g = tq+r
  \label{eqn:g_equals_tq_plus_r}
\end{align}
for some quotient $t \in \mathbb{F}_q$. Here
\begin{align*}
  r = r_0 + r_1 2^{64} + r_2 2^{128} + r_3 2^{192} \ \ \ \ \ \text{ where } r_i \in \{0,1,\ldots,2^{64}-1\}.
\end{align*}
As $q > 2^{254}$, the maximum value of $t$ required to satisfy this equation is $2^{392-254} = 2^{138}$. So the quotient requires only three 64-bit limbs in its reduced representation.
\begin{align*}
  t = t_0 + t_1 2^{64} + t_2 2^{128}\ \ \ \ \ \text{ where } t_i \in \{0,1,\ldots,2^{64}-1\}.
\end{align*}
The prime $q$ will have four 64-bit limbs.
\begin{align*}
  q = q_0 + q_1 2^{64} + q_2 2^{128} + q_3 2^{192} \ \ \ \ \ \text{ where } q_i \in \{0,1,\ldots,2^{64}-1\}.
\end{align*}

The main challenge in checking (\ref{eqn:g_equals_tq_plus_r}) is the mismatch in the representations of the LHS and the RHS. On the LHS, $g$ has an unreduced representation with four limbs each occupying upto 205 bits.
\begin{align*}
   g = h_0 + h_1 2^{64} + h_2 2^{128} + h_3 2^{192}.
\end{align*}
On the RHS, $tq+r$ has an unreduced representation with six limbs each occupying upto 130 bits.
\begin{align*}
  tq+r  =&\ t_0q_0+r_0 + (t_0q_1 + t_1q_0 + r_1) 2^{64} + (t_0q_2 + t_1q_1 + t_2q_0 + r_2) 2^{128}\\
  & + (t_0q_3 + t_1q_2 + t_2q_1 + r_3) 2^{192}+ (t_1q_3 + t_2q_2) 2^{256} + t_2q_3 2^{320}.
\end{align*}
One way to check equality in (\ref{eqn:g_equals_tq_plus_r}) is to convert both $g$ and $tq+r$ to their reduced representations and check that these representations are equal. This would require range checks on the limbs of $g$ and $tq+r$. The number of boolean variables required for each range check will be equal to the bitwidth of the corresponding limb. Each boolean variable requires one constraint of the form $b(b-1) = 0$ in the R1CS system.

We could reduce the number of range checks by checking that $g-tq-r$ equals zero. This is the approach used in \texttt{circom-ecdsa}. Consider the following argument assuming that $g-tq-r = 0$.
\begin{enumerate}
  \item $h_0-t_0q_0-r_0$ contains the 64 least significant bits of $g-tq-r$, i.e.~bits 0 to 63. These bits must all be zero. So $h_0-t_0q_0-r_0$ must be a multiple of $2^{64}$. 
  \item Let $y_0 = \frac{h_0-t_0q_0-r_0}{2^{64}}$. This represents the \textit{carry} into the $2^{64}$ limb.
  \item $y_0+h_1-t_0q_1-t_1q_0-r_1$ contains the \textit{next} 64 least significant bits of $g-tq-r$, i.e.~bits 64 to 127. These bits must also all be zero. So $y_0+h_1-t_0q_1-t_1q_0-r_1$ must be a multiple of $2^{64}$
  \item Let $y_1 = \frac{y_0+h_1-t_0q_1-t_1q_0-r_1}{2^{64}}$. This represents the carry into the $2^{128}$ limb.
  \item $y_1+h_2-t_0q_2-t_1q_1-t_2q_0-r_2$ contains the next 64 least significant bits of $g-tq-r$, i.e.~bits 128 to 191. These bits must also all be zero. So $y_1+h_2-t_0q_2-t_1q_1-t_2q_0-r_2$ must be a multiple of $2^{64}$
  \item Let $y_2 = \frac{y_1+h_2-t_0q_2-t_1q_1-t_2q_0-r_2}{2^{64}}$. This represents the carry into the $2^{192}$ limb.
  \item $y_2+h_3-t_0q_3-t_1q_2-t_2q_1-r_3$ contains the next 64 least significant bits of $g-tq-r$, i.e.~bits 192 to 255. These bits must also all be zero. So $y_2+h_3-t_0q_3-t_1q_2-t_2q_1-r_3$ must be a multiple of $2^{64}$
  \item Let $y_3 = \frac{y_2+h_3-t_0q_3-t_1q_2-t_2q_1-r_3}{2^{64}}$. This represents the carry into the $2^{256}$ limb.
  \item $y_3-t_1q_3-t_2q_2$ contains the next 64 least significant bits of $g-tq-r$, i.e.~bits 256 to 319. These bits must also all be zero. So $y_3-t_1q_3-t_2q_2$ must be a multiple of $2^{64}$
  \item Let $y_4 = \frac{y_3-t_1q_3-t_2q_2}{2^{64}}$. This represents the carry into the $2^{320}$ limb.
  \item $y_4-t_2q_3$ contains the remaining 72 least significant bits of $g-tq-r$, i.e.~bits 320 to 391. Recall that $g$ is bounded by $2^{392}$. These bits must also all be zero. So $y_4-t_2q_3$ \textit{must be zero}.
\end{enumerate}

Note that the limbs of $g-tq-r$ can have negative values, i.e. they can experience underflows during the subtraction operation. We use the convention that $x \in \mathbb{F}_n$ is \textit{negative} if $x > \frac{n-1}{2}$.  For example, $h_0-t_0q_0-r_0$ can be a negative multiple of $2^{64}$.

As the $h_i$'s have a maximum bitwidth of 205, the following terms (the unreduced limbs of $g-tq-r$) lie in the range $\{-2^{205}+1,\ldots, 2^{205}-1\}$.
\begin{align*}
  & h_0 - t_0q_0-r_0,\\
  & h_1 - t_0q_1-t_1q_0-r_1,\\
  & h_2 - t_0q_2-t_1q_1-t_2q_0-r_2,\\
  & h_3 -  t_0q_3-t_1q_2-t_2q_1-r_3,\\
  & - t_1q_3-t_2q_2,\\
  & - t_2q_3.
\end{align*}
Both the upper and lower ends of the ranges are conservative but we keep these values for convenience.

The procedure for checking $g-tq-r =0$ involves the addition of multiple terms some of which can be negative. Furthermore, the addition will be performed in $\mathbb{F}_n$. A set of terms sum to zero in $\mathbb{F}_n$ may not sum to zero in $\mathbb{F}_q$. To ensure that they do sum to zero in $\mathbb{F}_q$, we should ensure that the bitwidths of the partial sums does not exceed the capacity of $\mathbb{F}_n$.

The bitwidths of the terms $h_i, q_i, t_i, r_i$ will be known due to range checks. The carries $y_0,y_1,\ldots,y_4$ will be provided as non-deterministic advice to the arithmetic circuit. Instead of calculating $y_0$ as $\frac{h_0-t_0q_0-r_0}{2^{64}}$, we will check that $2^{64}y_0 = h_0-t_0q_0-r_0$ in the field $\mathbb{F}_n$. We need to apply range checks on the $y_i$'s to ensure that adding them will not exceed the capacity of $\mathbb{F}_n$.

\begin{itemize}
  \item As $h_0 - t_0q_0-r_0$ is in the range $\{-2^{205}+1,\ldots, 2^{205}-1\}$, $y_0$ can be checked to be in the range $\{-2^{141}+1,\ldots, 2^{141}-1\}$. In the arithmetic circuit, this is accomplished by checking that $y_0 + 2^{141}$ is in the range $\{0,1,\ldots,2^{142}-1\}$.
  \item $y_0+h_1-t_0q_1-t_1q_0-r_1$ is in the range $\{-2^{206}+1,\ldots, 2^{206}-1\}$. Since $y_1 = \frac{y_0+h_1-t_0q_1-t_1q_0-r_1}{2^{64}}$, we can check that $y_1$ is in the range $\{-2^{142}+1,\ldots, 2^{142}-1\}$. In the arithmetic circuit, this is accomplished by checking that $y_1 + 2^{142}$ is in the range $\{0,1,\ldots,2^{143}-1\}$.
  \item $y_1+h_2-t_0q_2-t_1q_1-t_2q_0-r_2$ is in the range $\{-2^{206}+1,\ldots, 2^{206}-1\}$. Since $y_2 = \frac{y_1+h_2-t_0q_2-t_1q_1-t_2q_0-r_2}{2^{64}}$, we can check that $y_2$ is in the range $\{-2^{142}+1,\ldots, 2^{142}-1\}$. In the arithmetic circuit, this is accomplished by checking that $y_2 + 2^{142}$ is in the range $\{0,1,\ldots,2^{143}-1\}$.
  \item $y_2+h_3-t_0q_3-t_1q_2-t_2q_1-r_3$ is in the range $\{-2^{206}+1,\ldots, 2^{206}-1\}$. Since $y_3 = \frac{y_2+h_3-t_0q_3-t_1q_2-t_2q_1-r_3}{2^{64}}$, we can check that $y_3$ is in the range $\{-2^{142}+1,\ldots, 2^{142}-1\}$. In the arithmetic circuit, this is accomplished by checking that $y_3 + 2^{142}$ is in the range $\{0,1,\ldots,2^{143}-1\}$.
  \item $y_3-t_1q_3-t_2q_2$ is (conservatively) in the range $\{-2^{206}+1,\ldots, 2^{206}-1\}$. Since $y_4 = \frac{y_3-t_1q_3-t_2q_2}{2^{64}}$, we can check that $y_4$ is in the range $\{-2^{142}+1,\ldots, 2^{142}-1\}$. In the arithmetic circuit, this is accomplished by checking that $y_4 + 2^{142}$ is in the range $\{0,1,\ldots,2^{143}-1\}$.
\end{itemize}
To simplify the arithmetic circuit logic (at least initially), the range checks on all the $y_i$'s can be for the range $\{-2^{142}+1,\ldots, 2^{142}-1\}$. Later the range check on $y_4$ can be for the smaller range $\{-2^{141}+1,\ldots,2^{141}-1\}$ and on $y_4$ can be for the smaller range $\{-2^{79}+1,\ldots,2^{79}-1\}$.
\end{document}
